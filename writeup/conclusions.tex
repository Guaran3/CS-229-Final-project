\section{Conclusions}

We developed an algorithm that predicts latent geographical
coordinates based on observed WiFi data by generating likelihood
models based on locally weighted sums of Gaussian distributions
centered at historical observations.
Our algorithm was able to deduce the general structure of the
Packard's layout, including the long hallway and the large lobby area.
From these results we observed, we conclude that the algorithm can
accurately localize a mobile device.
There remains much room for future improvement. For example, further testing
with various bandwidths and threshold levels will be
needed to find a better tradeoff between bias and variance. In future
studies, we hope to include timestamps into the design matrix and
develop constraints based on time data.
Once classification can be done with higher degrees of confidence and accuracy,
implementation in a mobile environment would be the next step. Moving the operations
from Matlab to C or Python could increase our computational speed to facilitate
future integration with a mobile application.

