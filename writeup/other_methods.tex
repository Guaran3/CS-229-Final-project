\section{Comparison of localization techniques}
Throughout this project we tried several different techniques to get more accurate
localization using WiFi.  The two that are representative of these methods are 
Gaussian Process Latent Variable Models and WiFi triangulation.  
\subsection{GP-LVM}
The currently most popular and arguably most state of the art technique
is the Gaussian Process Latent variable model.  In this model, they use
a similar kernel to our in terms of $w$, a squared exponential weighting 
for WiFi points.  However, they also assign to each WiFi reading a latent variable
tied to the time the sample was taken.  This allows them to make the following 
assumptions: successive positions in time have a proximity component, the change in 
position is likely parallel to other similar reading transitions, and orientation
between successive points is relatively low.  (this follows from the fact that 
most hallways are straight corridors. Their final constraint, again like ours, is 
to assume that similar signal strength readings possess similar coordinates in space.
While we began our testing using this model, it was not very effective, since we 
did not have the temporal data that we needed.  Data collection for a single phone
occurred at most once every 15 seconds, and more commonly was on the order of several 
minutes.  This essentially zeroed out any calculations involving the latent variables,
making this method ineffective.

\subsection{Triangulation}
This is certainly the most naive approach when attempting localization. Essentially, 
one considers signal strengths to be distances from an Access point with an exponentially 
decaying signal strength and random noise factor, $\varepsilon$. When we tried this, we used two steps:
We began by using a portion of the data to try to localize the APs nearby, and from there used the
rest of the data in for trying to find any error terms.  This was even less effective than the 
above method, due to the faulty assumption that signal strength decays uniformly as a function
of radius.  As mentioned previously, walls, furniture, and even other people in thte building can 
effect the signal strength of a particular access point, which makes it virtually impossible to
use this method without significant error.
