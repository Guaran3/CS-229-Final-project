\section{Description of Problem}

There are several key challenges associated with localization, and 
in particular doing so with the limitations imposed by using smart
phone data. These include data collection, power usage of the phones,
and the challenge of localization itself.  All three of these aspects
add to the difficulty in this problem. 

\subsection{Data Collection}
One obstacle to cell phone localization is in data collection.  
It is certainly desirable to take thousands of scans of a particular
building, and map it to a ground truth map; however, the time and 
cost of this method is prohibitively expensive.  Instead, an intelligent
method for extracting the most information with the least cost must be 
found.  Instead of this, we chose to use existing smart phone applications
to assist us in data collection.  Working with a company that is 
developing other location based software, we would piggyback data
that we would need along with theirs.  This means that anyone with 
this application installed will be collecting data, and will not
drain batteries more than would ordinarily occur.  While this certainly
makes it simple to collect huge amounts of data in a relatively short amount
of time, it creates another problem: which data do we need for a particular building.
This problem was beyond the scope of our current goals, and for this 
project, data was simply narrowed down by hand to a specific building as 
necessary.  

\subsection{Phone Limitations}
Another obstacle to designing an algorithm for smart phones are the limitations
of the phones themselves.  As mentioned above, one major limitation 
for phones is battery life.  This is in fact the single largest obstacle
to designing more complex algorithms, as we cannot insist that the user constantly 
be scanning wifi networks in order to localize, as that would be an enourmous strain 
on battery consumption.  One aspect of phones that we could not account for at all
was dealing with the wide variety of phones and corresponding hardware, putting 
limits on how much we could trust the data given to us. Finally, the computing and 
memory requirements have to be taken into consideration.  While it is true that 
smart phones are highly capable machines, the users themselves don't want an
application that takes gigabytes of data just to improve accuracy in localization.
Similarly, in order for localization to be useful, it must be relatively fast 
(ie < 1-2 seconds).  This also limits how we are able to process incoming data.

\subsection{Localization Challenges}
The key challenge of localization is overcoming the unpredictability
of WiFi signal propagation through indoor environments. The data
distribution may vary based on changes in temperature and humidity, as
well as the position of moving obstacles, such as people walking
throughout the building. This uncertainty makes it difficult to
generate accurate estimates of signal strength measurements. Thus, the
bulk of research in this area focuses on refining the location
likelihood models from the available data collected in the
environment.

\subsection{Approach}
In order to resolve these challenges, we present a machine learning
based algorithm for localizing a mobile device using a locally
weighted regression to map high-dimensional data to a likelihood
function in a low-dimensional latent space. In our context, the
high-dimensional data represents signal strength for all WiFi access
points in the indoor environment, and the low-dimensional space
corresponds to the geographical coordinates of the device's location.
Our technique considers that signal strength correlates with physical
location. Observations with similar signal strength measurements are
likely to be close to each other. This constraint is important when
dealing with data from close loops, where the person visits a location
at two different points in time.
The data we collected consists of signal strength measurements
annotated with GPS readings and error estimates. We collected data
sets via the method described above, and chose Packard building to model,
as it has a relatively simple layout and we had the most data for this building.
For calculation, we offloaded this to the server for two reasons in particular.
The first, and most obvious, is that local regression is non-parametric and 
thus we would need to send a phone large amounts of data in order to process it
onboard.  Second, this allows us to use the significantly higher processing power
of the servers.
